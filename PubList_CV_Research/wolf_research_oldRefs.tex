
%%%%%%%%%%%%%%%%%%%%%%%%%%%%%%%%%%%%%%%%%%
%
%  Note: {Tichy_pn} has to be changed
%
%%%%%%%%%%%%%%%%%%%%%%%%%%%%%%%%%%%%%%%%%%


\documentclass[11pt]{article}

%\hfuzz 10pt
%\font\bigbf=cmb10 scaled\magstep2
%\font\smaller=cmr9


% From Phil
\setlength{\textwidth}{7in}
\setlength{\textheight}{9.0in}
\setlength{\hoffset}{-0.275in}
\setlength{\oddsidemargin}{0in}
\setlength{\evensidemargin}{0in}
\setlength{\topmargin}{0in}
\setlength{\voffset}{0.5in}  % for RHAT 6.2 latex  
\setlength{\headheight}{0in}
\setlength{\headsep}{0in}


\begin{document}

\centerline{\bf \large Wolfgang Tichy's Research Interests}
\bigskip


My research interests span a broad range of topics including
projects in 
numerical relativity (\S 1),
post-Newtonian theory (\S 2),
gravitational waves (\S 3),
semiclassical relativity (\S 4).
% Although I have several ideas for future
% research in these same topics, I also eagerly look forward to working in
% new, and perhaps different, areas of research in General Relativity,
% theoretical astrophysics, cosmology and/or numerical relativity.
I have several ideas for future research in all these topics.
Yet at the moment most of my research is in the area of numerical relativity.


\medskip


\bigskip 
\bigskip 
\noindent
{\bf 1. Numerical Relativity}

\bigskip 
\noindent
{\it a. Post-Newtonian initial data for black hole mergers}
\medskip

Mergers of two black holes both with masses of 
$\sim 10-100M_{\odot}$ will be
observable by the ground based gravitational wave detectors. 
These systems are highly relativistic once they enter the sensitive 
frequency band ($\sim 50 - 200 \, {\rm Hz}$) of the detector. Hence
in order to predict their orbits and waveforms, fully nonlinear numerical
simulations should be used instead of standard post-Newtonian calculations.
Of course such numerical simulations must begin by specifying initial data
and the whole simulation will only be astrophysically relevant if
astrophysically realistic initial data are used.
Now, in general relativity the initial data must fulfill constraint equations,
so that only part of the data are freely specifiable, the rest is then
determined by solving the constraint 
equations (for a review see e.g. \cite{Cook_LivRev}). 
A lot of the work in constructing initial data has focused on 
approaches that pick the freely specifiable part of the data, with the aim
of simplifying the constraint equations, rather
than picking astrophysically realistic initial data 
(see e.g. \cite{Bowen_York,Cook_EffecPot}).
It is therefore important to construct initial data, which accurately
represent astrophysical systems 
such as two black holes orbiting each other.

Together with Bernd Bruegmann, Manuela Campanelli and Peter Diener,
I am working on a project to generate initial data
for two inspiraling black holes from post-Newtonian expressions. 
The idea is that even though post-Newtonian theory may not be able to evolve
two black holes when they get close, it can still provide initial
data for fully nonlinear numerical simulations when we start
at a separation where post-Newtonian theory is valid.
Hence we needed expressions for the 3-metric and
extrinsic curvature in a convenient gauge. 
We decided to use post-Newtonian expressions computed in the ADM gauge
by P. Jaranowski and G. Sch\"afer \cite{Jara_Schafer}, who were
so kind to send us their expressions in a Mathematica file.
We use their expressions for the 3-metric and conjugate momentum,
up to post-Newtonian order $ (v/c)^5 $.
The ADM gauge has at least four advantages:
(i) we can easily find
expressions for 3-metric and extrinsic curvature,
(ii) unlike in the harmonic gauge no logarithmic divergences appear, 
(iii) up to $(v/c)^3$ the data can be written in puncture form
\cite{Punc}, which simplifies calculations, and
(iv) the trace of the extrinsic curvature vanishes up to order $(v/c)^6$,
so that we can set it to zero (if we go only up to order $ (v/c)^5$),
which can be used to decouple the Hamiltonian 
constraint equation from the momentum constraint equations.
In the ADM gauge the 3-metric is conformally flat up to order $ (v/c)^3$,
at order $ (v/c)^4$ deviations from conformal flatness enter. 
The extrinsic curvature up to order $ (v/c)^3$ is simply of 
Bowen-York form \cite{Bowen_York}, with correction terms of order $ (v/c)^5$.

One problem is that post-Newtonian theory in principle deals with point
particles and not black holes, and that the post-Newtonian data
are not valid near each particle. 
To overcome this problem we rewrite the post-Newtonian data
such that each post-Newtonian point particle is replaced by a 
puncture \cite{Punc}
coordinate singlarity. This amounts to adding post-Newtonian terms
that are higher than $(v/c)^5$.�As we are intersted in a post-Newtonian
accuracy of only $(v/c)^5$, this replacement does not change the data
as far as post-Newtonian theory up to $(v/c)^5$ is concerned.
The advantage however is that then the data look like Schwarzschild
black holes plus a perturbation near each particle. Furthermore the data
then closely resemble puncture data, which is advantageous in numerical
calculations.

Another problem with the post-Newtonian data is that the expressions for
the 3-metric and extrinsic curvature do not fulfill the constraint equations.
We therefore use the York-Lichnerowicz conformal 
decomposition \cite{York_Decomp} with the
post-Newtonian data used as the freely specifiable data 
and numerically solve for a new conformal factor $\Psi$ 
and the usual correction to the extrinsic curvature,
coming from a vector potential. The new extrinsic curvature and the 3-metric
multiplied by $\Psi^4$ are then guaranteed to fulfill the constraints.
The biggest numerical problem in this approach is to find a numerical scheme
which can deal with the divergences in the post-Newtonian data
at the center of each black hole. However as the data are very similar to
puncture data, we can conformally rescale the post-Newtonian data by
appropriate powers of the post-Newtonian conformal factor $\psi_{PN}$
to obtain a well behaved 3-metric. If we then make the Ansatz that
the new conformal factor $\Psi$ is the post-Newtonian
conformal factor $\psi_{PN}$ plus a finite correction $u$,
we arrive at elliptic equations which can be solved numerically.

The question then arises how realistic the corrected constraint fulfilling
data are.  One obvious problem is that we have changed the post-Newtonian
data in order to satisfy the constraints. Moreover since we solve elliptic
equations this modification is non-local, i.e. the large errors in the data
before the elliptic solve near the black holes will change the solution
everywhere, even in regions where the constraints are well satisfied before
the elliptic solve. In turns out that this change is on the same order of
magnitude as the highest post-Newtonian terms we have included, so that
after enforcing the constraints we have likely lost some of the
post-Newtonian characteristics, which makes the data appealing in the first
place. In particular the ADM and apparent horzon masses of the data always
increase due to the elliptic solve. We have however 
found a way to compensate for this increase by decreasing the post-Newtonian 
conformal factor $\psi_{PN}$ before the solve 
by a term of order $ (v/c)^4$ \cite{Tichy_PNdata}.
The result of this modification is that after the elliptic solve, the data
have the same ADM and apparent horzon masses as the pure post-Newtonian
data. In addition the difference between pure and solved data 
in the region where post-Newtonian theory is valid,
is now smaller than the highest post-Newtonian corrections included. Hence
we can find constraint satisfying data, which incorporate post-Newtonian
features.

\bigskip   
\noindent
{ \it Future outlook}
\medskip

The next step is to check by how much the initial data differ from a tidally
deformed black hole in the region near each the horizon.
Furthermore I would like to compare our data with initial data from the 
puncture approach and check the influence of the
non-conformally flat post-Newtonian terms, after all 
up to order $(v/c)^3$ our data are equal to puncture data.
After all these checks I want to 
use our data as the starting point for numerical evolutions and see
what waveforms we get. 




\bigskip
\noindent
{\it b. Hyperbolic formulations of Einstein's equations 
for black hole evolutions}
\medskip


\bigskip
\noindent
{\it c. Gauge conditions for binary black hole
puncture data based on an approximate helical Killing vector}
\medskip                          







\bigskip
\bigskip
\noindent
{\bf 2. Post-Newtonian theory}

\bigskip
\noindent
{\it Coordinate independent formulation of post-1-Newtonian theory } 
\medskip

In many problems
of interest in astrophysics it is
difficult or impossible to solve the full Einstein equations
of General Relativity analytically. For this reason the so called
post-Newtonian approximation has been 
developed \cite{Chandra,Chandra_cons,Chandra2PN}.
Usually, the equations of post-Newtonian Theory are given in
a specific coordinate system (or gauge). 
The gauge is chosen such that it simplifies the equations. For
example the coordinates are usually
chosen to be as closely Cartesian as possible.
The three most often used gauges (see e.g. \cite{Damour}) are: 
the harmonic gauge, the standard post-Newtonian gauge, and 
the conformally Cartesian gauge.
In each of these gauges the post-Newtonian equations take a different form.
This plethora of different gauges
has several disadvantages.
Often it is difficult to see if two calculations agree, simply because
they are done in different coordinate systems.
Even though there are several standard choices for the coordinates (e.g.
the harmonic gauge, the standard gauge, ...), it is sometimes better
to use different coordinates, which are better 
adapted to the problem at hand. 

In 1998 Professor \'E. Flanagan and I started investigating the possibility of
a gauge independent formulation of post-Newtonian General 
Relativity \cite{Tichy_pn}.  
We expanded the
metric and connection of General Relativity in terms of a formal parameter
$\epsilon \sim v/c$. 
The expansion coefficients at each order in
$\epsilon$ are coordinate independent tensor fields.  
We followed
earlier work by Dautcourt \cite{Dautcourt0,Dautcourt}, who 
derived the coordinate independent form of Newtonian gravity
(called Newton-Cartan theory).

The only assumptions we need to make in our expansions are 
(i) that the upper index metric starts at order $\epsilon^0$,
(ii) that it has signature $(0+++)$ at leading order
and (iii) that the contravariant
stress-energy tensor describing matter starts at 
order $\epsilon^4$.
Inserting these expansions into the standard equations of General Relativity
then yields two sets of equations.
At order $\epsilon^0 $ we were able to recover the Newton-Cartan
geometric formulation of Newtonian gravity, while
at order $\epsilon^2 $ we found the first post-Newtonian
corrections to Newton-Cartan Theory \cite{Tichy_pn}.
If the resulting equations are considered as the fundamental
equations of post-Newtonian theory, one finds
that the theory possesses a symmetric connection, but no spacetime metric
and that it is formulated solely in terms of geometric objects.
Its connection has Newtonian and post-Newtonian pieces,
which contain the dynamical degrees of freedom.
As in the Newton-Cartan case, the theory allows for a scalar function,
which can be interpreted as time.
% Furthermore the theory allows for a scalar function, which we 
% interpret as time.
We were then able to show from our equations 
that the usual coordinate-dependent equations of post-Newtonian gravity 
can be recovered when one specializes to asymptotically flat situations
and to appropriate classes of coordinates.



\bigskip
\noindent
{ \it Future outlook }  
\medskip



It should be possible to find higher than just
post-1-Newtonian corrections by expanding General Relativity to
higher orders in the formal parameter $\epsilon$. 
Also, it would be interesting to find fully covariant definitions of the
quantities that are conserved (e.g. energy and angular momentum) at
post-1-Newtonian order. 
In the
usual coordinate formulation of post-Newtonian theory conserved quantities
are defined as volume or surface integrals over suitable stress-energy
tensors. The problem is that our theory does not posses a metric, so that
one encounters difficulties in defining volume integrals, since no natural
volume element is available. One way around this could be to find exact
3-forms and to use Stokes' law to find conserved integrals.  Such 3-forms
might be constructed from the stress-energy tensor and approximate Killing
vectors. Another possible route for finding conserved quantities might be
to introduce a $3+1$ split in the theory and use the resulting
spatial metric to define volume integrals. Such a $3+1$ 
formulation is easily achieved since we
already have a preferred time function available.


\bigskip
\bigskip
\noindent
{\bf 3. Gravitational Waves}

 
\bigskip
\noindent
{a. \it Post-Newtonian gravitational waveforms  }

\medskip

The inspiral and merger of compact binaries is thought to be an important
source of gravitational waves which could be detected by the LIGO, VIRGO 
or GEO detectors. 
The detection of gravitational waves from such an inspiral
using matched filtering depends crucially on the availability of
accurate template waveforms. 
In the case of a nearly circular inspiral of two point masses the expected
gravitational wave signal has the form of a chirp, i.e., a roughly
sinusoidal signal with gradually increasing amplitude and frequency.
If such a signal is to be detected by matched filtering,
high accuracy is needed in the templates, in particular the phase of the
template must closely match the phase of the actual signal.
Such inspiral templates of the waveform and its phase have been 
calculated up to 
2.5 post-Newtonian order \cite{Blan-I-Will-Wis,Blanchet}.
The phase is derived from the energy balance equation
which basically states that the energy loss rate 
is equal to the gravitational wave luminosity of the binary.

\'E. Flanagan, E. Poisson and I \cite{Tichy_gwnum}
have investigated if the accuracy of the templates' phasing can be
improved by solving the post-Newtonian energy balance equation exactly
numerically, rather than (as is normally done) solving the energy balance
equation analytically within the post-Newtonian perturbative expansion.
By specializing to the limit of a small mass ratio, we found evidence that
there is no gain in accuracy. This result is disappointing, but constitutes
useful information from the point of view of generating template banks for
inspiral searches: there is no motivation in terms of increased event rate
to solve numerically for the wave's phasing.




\bigskip
\noindent
{b. \it Work in progress on computing the radiation reaction force
        for point particles in Kerr spacetime}
\medskip

The inspiral of compact objects of $1-10M_{\odot}$ 
into supermassive black holes of $\sim 10^{6}M_{\odot}$ is an
important source of gravitational waves which could be detected by the
planned space-based LISA detector \cite{Finn_Thorne}.
Such systems go through roughly $10^{5}$ orbital cycles while they are in
LISA's sensitive frequency band of $\sim 10^{-3}-1$ Hz. Hence, 
in order to find accurate gravitational wave templates, we
need to compute the inspiral orbit to high accuracy (of order $10^{-5}$).
Since the mass ratio is small,
the compact object is well
approximated by a test-particle.
Now, in order to find the orbit of a test-particle spiraling into a 
black hole, one in principle has to compute the radiation reaction
force. Yet, even though an analytic
formula for the radiation reaction force has been given
-\cite{Mino_Sasaki,Quinn_Wald}, it has been impossible so far 
to evaluate it in practice in Kerr spacetime due
to the presence of tail terms. 
For this reason we focus on a method involving conservation laws and
make use of the fact that the radiation reaction
timescale is large compared to the dynamical timescale, if the mass ratio is  
small. 
In this case the test-particle orbit can be approximated as a sequence
of geodesic orbits, where the constants of the motion
describing a geodesic slowly change on a radiation reaction timescale. This
approach has already been successfully implemented for general orbits around
Schwarzschild black holes, equatorial orbits around Kerr
black holes \cite{Kennefick}, 
and also for inclined circular orbits around Kerr
black holes \cite{Hughes}. Yet, in the astrophysically relevant case of
generic inspiral orbits around rotating (Kerr) black holes, there is a
problem. It is not yet known how to compute the adiabatic evolution of the
Carter constant $C$ due to radiation reaction. On the other hand, the
evolution of the
particle's energy $E$ and $z$-component of angular momentum $L_z$,  can be
computed from the energy and $z$-component of angular momentum fluxes carried
by the gravitational waves to null infinity and down the black hole horizon.



In 2000 Professor \'E. Flanagan 
and I have started exploring the feasibility of
possible methods for computing the evolution of the Carter constant.
We first investigated a particular approach which was suggested
several years ago by Sam Finn and more recently by John Friedman.
It involves computing the $x$-
and $y$-components of angular momentum
$J_x$ and $J_y$ at future null infinity.             	 
The approach envisages (i) computing for a given geodesic orbit the flux of
$J_x$ and $J_y$ at future null infinity, (ii)        	 
computing from the time-averaged linear metric perturbation the static
$J_x$ and $J_y$ as functions of                      	 
$C$, $E$, $L_z$ and appropriate parameters describing the black hole,
and (iii) solving for $\dot{C}$                      	 
by equating the angular momentum flux of step (i)    	 
to the time derivative the angular momentum of step (ii).
The first potential problem with this method is the fact that $x$- and
$y$-components of angular momentum and angular momentum fluxes are not
necessarily well defined.  However, we find \cite{Tichy_MGM9, Tichy_LISA} 
that the
ambiguities in these quantities arise at higher order in the adiabatic
and test-particle expansions than the leading order terms relevant to
this computation.  Therefore, this first problem can be circumvented.
A more serious problem is that we have shown \cite{Tichy_LISA} that the 
$x$- and $y$-components of the flux vanish identically, 
when the $z$-axis is taken
to point in the direction of the total angular momentum of the system,
so that no useful information is obtained from $x$- and $y$-components
of angular momentum.  



\bigskip   
\noindent
{ \it Future outlook}
\medskip

An alternative approach which we are currently exploring is based on the
fact that for scalar fields on the Kerr background, there is a
conserved quantity associated with the Killing tensor used in the
definition of the Carter constant \cite{priv_Ashtekar}. 
It seems likely that this quantity and its fluxes can 
be used instead of angular momentum in the steps (i)-(iii) above to 
compute the evolution of the Carter constant in the case of scalar
radiation. 
Furthermore it is possible that an analogous conserved quantity
exists for linearized metric perturbations in Kerr, and we are exploring
this issue.  If such a conserved quantity does exist, it should be
possible to compute, from the gravitational waves at the black hole
horizon, a quantity allowing one to track the evolution of the Carter
constant.  
It might also be worthwhile to investigate 
the possibility of extending
the calculation to spinning test-particles.
This is interesting since particles with spin will significantly deviate
from a geodesic, even without radiation reaction, once the particle
comes close to the black hole. 





\bigskip
\bigskip
\noindent
{\bf 4. Semiclassical Relativity}

\bigskip
\noindent
{\it The stress-energy tensor of a massive scalar field }
\medskip

In semiclassical gravity, a classical metric is coupled to quantum fields.
This is achieved by replacing the stress-energy tensor by its expected value
in the Einstein equation. The resulting semiclassical Einstein equation is
usually postulated rather than derived as there is no complete theory of
quantum gravity from which it could be derived.
 There are several well-known difficulties associated
with the semiclassical theory.  First, there are difficulties
associated with the existence of unphysical, exponentially growing
``runaway'' solutions of the semiclassical Einstein equation, 
which have not yet
been completely resolved \cite{anec}.  The second difficulty,
which is the subject of a research project \cite{Tichy_stress} 
Professor \'E. Flanagan and I
started in 1997, is the non-uniqueness of the
expected stress-energy tensor on the right hand side of
semiclassical Einstein equation.

For a scalar field, several methods have been suggested for calculating
the expected stress energy tensor.  These include
(i) the ``point splitting'' algorithm \cite{WaldQFT}, (ii) the
deWitt-Schwinger expansion method \cite{BirDav}, and (iii) the in-in
effective action method \cite{Schwing,J1,CampVar}.  There is no
general agreement as to which method is correct.  For
example, it is claimed in Ref.\ \cite{WaldQFT} that the
deWitt-Schwinger method is invalid for a massive scalar field since it
does not have a regular limit as $m \to 0$, where $m$ is the mass.

As is well known, a theorem of Wald \cite{Waldaxioms,WaldQFT} plays a
crucial role in this field.  The theorem states that if one has two
different prescriptions for obtaining stress tensors from metrics and
from quantum states, and if these prescriptions obey a certain set of
physically-motivated axioms, then the two prescriptions must agree up
to local conserved curvature tensors. 
The conventional view had been that this ambiguity is just a two parameter
ambiguity, since there exist only two independent conserved local curvature
tensors with the correct dimensions for a stress-energy tensor. We found
that this argument, based on dimensional analysis, holds true only for
massless scalar fields.  In the case of a massive scalar field the above
argument fails, since there is a preferred mass scale present, namely the
mass $m$ of the field.  
Using this mass scale we were able to construct infinitely many 
local conserved tensors by simply dividing by appropriate powers of 
$m$ \cite{Tichy_stress},
such that the tensors have the correct dimensions of a stress-energy tensor.
The ambiguity in the stress-energy tensor allowed by the Wald
axioms is therefore much worse in the massive case than in the
massless case.  It is an infinite parameter ambiguity 
rather than a two parameter
ambiguity.  Of course, it is still possible that the various
conventional calculational methods still agree to within the two
parameter ambiguity \cite{Moretti}.  However,
there is no guarantee that this should be the case.  



Spacetimes which are linear perturbations off Minkowski spacetime
form a useful testbed in which to probe these issues.
In order to get an explicit example, we have calculated
the expected stress-energy tensor in the incoming vacuum state 
for a massive scalar field in such
spacetimes, using the in-in effective action formalism
\cite{Schwing,J1,CampVar}.  
The stress-energy tensor we have found \cite{Tichy_stress}
is causal, as it must be, and reduces to the known result of the
massless case \cite{Horowitz,J2} in the limit $m \to 0$.
Furthermore it is not a smooth function of $m^2$ at $m^2 =0$.
The calculational method we used also automatically yields two
undetermined parameters, so the result explicitly exhibits a two parameter
ambiguity, just as in the massless case \cite{Horowitz},
even though this is not guaranteed by the Wald axioms.

\bigskip
\noindent
{ \it Future outlook }  
\medskip

After our paper \cite{Tichy_stress} had been published, 
Salcedo \cite{Salcedo} showed that within the context of perturbative 
quantum field theory (such as the effective action formalism)
no more than a two
parameter ambiguity can arise in the stress-energy tensor of a massive
scalar field. This means that all methods using
perturbative quantum field theory must
agree up to the two parameter ambiguity. Yet, it is still possible that a
larger ambiguity will be found if non-perturbative effects are taken into
account. This raises the important question if there are additional axioms
which would reduce the ambiguity in the stress-energy tensor.










\clearpage

\begin{thebibliography}{99}


%%% NumRel %%%%%

\bibitem{Cook_LivRev}
G. B. Cook, Living Reviews in Relativity,
http://www.livingreviews.org/Articles/Volume3/2000-5cook (2000)

\bibitem{Bowen_York}
J. M. Bowen and J. W. York Jr., Phys. Rev. D {\bf 21}, 2047 (1980)

\bibitem{Cook_EffecPot}
G. B. Cook, Phys. Rev. D {\bf 50},5025 (1994) 

\bibitem{Jara_Schafer}
P. Jaranowski and G. Sch\"afer, Phys. Rev. D {\bf 57}, 7274 (1998)

\bibitem{York_Decomp}
J. W. York Jr., J. Math. Phys. {\bf 14}(4), 456 (1973)

\bibitem{Punc}
S. Brandt and B. Bruegmann,
Phys. Rev. Lett. {\bf 78}, 3606 (1997) 

\bibitem{Tichy_PNdata}
W. Tichy, B. Bruegmann, M. Campanelli and P. Diener,
Phys. Rev. {\bf D 67}, 064008 (2003), gr-qc/0207011



%%%% PN-Theory %%%%

\bibitem{Chandra}
S. Chandrasekhar, Astrophys. J. {\bf 142}, 1488 (1965)

\bibitem{Chandra_cons}
S. Chandrasekhar, Astrophys. J. {\bf 158}, 45 (1969)

\bibitem{Chandra2PN}
S. Chandrasekhar and Y. Nutku, Astrophys. J. {\bf 158}, 55 (1969)

\bibitem{Damour}
T. Damour, M. S. Soffel and C. Xu,  Phys. Rev. D {\bf 43}, 3273 (1991)

\bibitem{Tichy_pn}
W. Tichy and \'E. \'E. Flanagan, in preparation

\bibitem{Dautcourt0}
G. Dautcourt, Acta Physica Polonica {\bf B21}, 755 (1990)

\bibitem{Dautcourt}
G. Dautcourt, Class. Quant. Grav. 14: A109-A118 (1997),
e-Print Archive: gr-qc/9610036


%%% GWs %%%%%

\bibitem{Blan-I-Will-Wis}
L. Blanchet, B. R. Iyer, C. M. Will, A. G. Wiseman,
Class. Quant. Grav. {\bf 13}, 575 (1996)

% \bibitem{Blan-Dam-Will-Wis}
% L. Blanchet, T. Damour, B. R. Iyer, C. M. Will, A. G. Wiseman,
% Phys. Rev. Lett. {\bf 74}, 3515 (1995)

\bibitem{Blanchet}
L. Blanchet, Phys. Rev. D {\bf 54}, 1417 (1996)

\bibitem{Tichy_gwnum}
W. Tichy, \'E. \'E. Flanagan and E. Poisson, 
Phys. Rev. D {\bf 61}, 104015 (2000) , gr-qc/9912075

\bibitem{Finn_Thorne}
L. S. Finn and K. S. Thorne, gr-qc/0007074

% \bibitem{Sigurdsson}
% S. Sigurdsson and M. J. Rees, 
% Mon. Not. R. Astron. Soc {\bf 284}, 318 (1997)

\bibitem{Mino_Sasaki}
Y. Mino, M. Sasaki and T. Tanaka,
Phys. Rev. D {\bf 55}, 3457 (1997)

\bibitem{Quinn_Wald}
T. C. Quinn and R. M. Wald,
Phys. Rev. D {\bf 56}, 3381 (1997)

\bibitem{Kennefick}
D. Kennefick, Phys. Rev. D {\bf 58}, 064012 (1998)

\bibitem{Hughes}
S. A. Hughes, Phys. Rev. D {\bf 61}, 084004 (2000) 

% \bibitem{BMS}
% R. Sachs, Phys. Rev. {\bf 128}, 2851 (1962)

% \bibitem{MTW}
% C. W. Misner, K. S. Thorne and J. A. Wheeler, {\it Gravitation},
% W. H. Freeman and Company, New York (1973)

\bibitem{Tichy_MGM9}
W. Tichy and \'E. \'E. Flanagan,
  edited by V.G. Gurzadyan, R.T. Jantzen and R. Ruffini,
  World Scientific, Singapore, 1622, (2002)

\bibitem{Tichy_LISA}
W. Tichy and \'E. \'E. Flanagan, Class. Quant. Grav. {\bf 18}, 3995 (2001)

\bibitem{priv_Ashtekar}
A. Ashtekar, private communication



%%% Semiclassical Relativity %%%%

\bibitem{anec}
\'E.\ \'E.\ Flanagan and R.\ M.\ Wald, Phys. Rev. D {\bf 54}, 6233 (1996)

\bibitem{Tichy_stress}
W. Tichy and \'E. \'E. Flanagan,
Phys. Rev. D {\bf 58}, 124007 (1998) , gr-qc/9807015

\bibitem{WaldQFT}
R.M.~Wald, {\it Quantum Field Theory in Curved
Spacetime and Black Hole Thermodynamics}, University of Chicago Press,
Chicago (1994).

\bibitem{BirDav}
N. D. Birell and P. C. W. Davis {\it Quantum fields in curved space},
Cambridge University Press, Cambridge (1982).

\bibitem{Schwing}
J. Schwinger, J. Math. Phys. {\bf 2}, 407 (1961).

\bibitem{J1}
R. D. Jordan, Phys. Rev. D {\bf 33}, 444 (1986).

\bibitem{CampVar}
A. Campos and E. Verdaguer, Phys. Rev. D {\bf 49}, 1861 (1994).

\bibitem{Waldaxioms}
R.\ M.\ Wald, Commun.\ Math.\ Phys.\ {\bf 54}, 1 (1977); Phys. Rev. D
{\bf 17}, 1477 (1978).

\bibitem{Moretti}
V. Moretti, Phys. Rev. D {\bf 56}, 7797 (1997), gr-qc/9805091

\bibitem{Horowitz}
G. T. Horowitz, Phys. Rev. D {\bf 21}, 1445 (1979).

\bibitem{J2}
R. D. Jordan, Phys. Rev. D {\bf 36}, 3593 (1987).

\bibitem{Salcedo}
L. L. Salcedo, Phys. Rev. D {\bf 60}, 107502 (1999)




\end{thebibliography}



\end{document}


