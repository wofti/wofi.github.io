%%%%%%%%%%%%%%%%%%%%%%%%%%%%%%%%%%%%%%%%%%
%
%  Note: {Tichy03} has to be changed
%	 {priv_Ashtekar} is still undefined
%
%%%%%%%%%%%%%%%%%%%%%%%%%%%%%%%%%%%%%%%%%%

\documentclass[11pt]{article}

%\hfuzz 10pt
%\font\bigbf=cmb10 scaled\magstep2
%\font\smaller=cmr9


% From Phil
\setlength{\textwidth}{17.1cm}
\setlength{\textheight}{24.2cm}
\setlength{\hoffset}{-0.2cm}
\setlength{\oddsidemargin}{0cm}
\setlength{\evensidemargin}{0cm}
\setlength{\topmargin}{0cm}
\setlength{\voffset}{-0.8cm} 
\setlength{\headheight}{0cm}
\setlength{\headsep}{0cm}

\usepackage{setspace}

\begin{document}
\begin{spacing}{0.95} 


\centerline{\bf \Large Research Interests: Wolfgang Tichy}
\bigskip

My main expertise is in numerical simulations of the Einstein equations in
the fully non-linear regime. In particular, I am working on simulations of
binary black hole and binary neutron star systems. 
I have experience in both the time evolution of
such systems and also in the construction of realistic initial data.
The accurate modeling of the inspiral and merger phase of black hole 
or neutron star binaries is one of the
major challenges in gravitational wave physics today.
Due to advances in supercomputer technology,
new formulations of the evolution equations,
and improved numerical methods it now possible to simulate
such binaries from the late inspiral stage all the way to 
the final merged state. The results of
these simulations are currently being used to extract and analyze signals of
binary inspirals and mergers from data acquired by
interferometric gravitational wave detectors such as LIGO, GEO and VIRGO.
%and they will also be used for the planned LISA.
Apart from being a pressing problem from the data analysis viewpoint, 
the merger of compact objects is also of fundamental importance,
as it probes the fully non-linear regime of Einstein's equations.
Furthermore, the analysis neutron star mergers can yields information
about the equation of state of matter at nuclear density.
These different aspects make it very
exciting for me to work in this field.

In addition, my research interests also span a broad range of other topics
in relativity and astrophysics including post-Newtonian theory,
gravitational waves, radiation reaction, and semiclassical relativity. In my
opinion it is very important to have a broad range of knowledge and
interests.  This ``big picture'' view facilitates cooperation with
researchers in other areas of science.




\medskip
\bigskip
\centerline{\bf \large Current and Past Research}
\medskip % \bigskip


\noindent
{\bf Initial data for black hole inspirals }
\smallskip

Mergers of two black holes with masses of $\sim 10-100M_{\odot}$ will be
observable by the ground based gravitational wave detectors.  In order to
predict their highly relativistic orbits and waveforms, fully nonlinear
numerical simulations should be used. All such numerical simulations must
begin by specifying initial data. I am very interested in constructing
initial data that accurately represent astrophysical systems such as two
black holes orbiting each other, because only then will a simulation be
astrophysically relevant. Together with B. Br\"ugmann, M. Campanelli and P.
Diener, I have worked on a project to generate post-Newtonian based initial
data for two inspiraling black holes. The idea is that even though
post-Newtonian theory may not be able to evolve two black holes when they
get close, it can still provide initial data for fully nonlinear numerical
simulations, if we start at a separation where post-Newtonian theory is
valid.
% We use post-Newtonian
% equations computed by P. Jaranowski and G. Sch\"afer as a starting point.
However, before this data can be used in general relativistic simulations we
have to modify it, since post-Newtonian theory in principle deals with point
particles and not black holes. Also, the pure post-Newtonian data do not
fulfill the constraint equations of General Relativity. To overcome these
problems we first resum the post-Newtonian expansion in such a way that the
data do contain black holes. In a second step we then project the data onto
the solution manifold of General Relativity by solving the constraint
equations~\cite{Tichy02} using the Cactus code. The resulting data do
satisfy the constraint equations of General Relativity, and incorporate
post-Newtonian features in the sense that they are close to the original
post-Newtonian data.



Post-Newtonian calculations predict that black hole binaries are expected to
move on quasi-circular orbits with a slowly shrinking radius, i.e.\ the
binary system is in quasi-equilibrium. Together with B. Br\"ugmann and P.
Laguna, I have started to investigate how to find coordinate systems which
corotate with the two orbiting black holes. Such coordinate systems have the
advantage that the rapid circular motion of the two black holes is
transformed away so that one has to simulate only the slower drift of the
holes toward each other. It is hoped that then the numerical simulations
will be more accurate and stable. 
% Furthermore techniques to excise the black
% hole singularity from the computational domain simplify if the black holes
% do not move through the domain. 
In order to find such corotating coordinates
for the initial data, we use several conditions which hold if the orbital
timescale is much shorter than the inspiral timescale. In particular, we use
necessary conditions for the existence of an approximate helical Killing
vector, such as equality of Komar and ADM mass. As a first step we have
applied these ideas~\cite{Tichy03a} to standard puncture initial data, which
are similar to, but simpler than, the post-Newtonian based initial data
discussed above. We have also been able to use our conditions to construct
sequences of quasi-equilibrium puncture initial data~\cite{Tichy:2003qi}.
% We have then numerically computed the magnitudes of the time derivatives 
% of different metric components. The result is that the 3-metric evolves on a
% timescale larger than the orbital timescale. However, some components of
% the extrinsic curvature still evolve on the orbital timescale.
% This means that we have been only partially succuessful in the construction
% of quasi-equilibrium data.


%\medskip % \bigskip
%\noindent
%{\it Extreme mass ratio limit of puncture data }
%\medskip

All numerical calculations described so far were carried out on finite
spatial domains using finite differencing techniques. Recently M. Ansorg, B.
Br\"ugmann and I have implemented a new numerical method for the computation
of puncture data. It involves a coordinate transformation, which
compactifies all of space into a finite box.
% , with the punctures sitting on two edges. 
In these coordinates we are able to use a spectral method to
numerically solve the initial data equations. The result is much more
accurate and requires far less computational resources than the one
obtained using conventional finite differencing techniques.
We are working on using this accuracy to investigate puncture data in the
small mass ratio limit. The aim is to compare puncture data in this limit
with the known analytic solution of a point particle in circular orbit
around a black hole. In this way it will be possible to assess the 
astrophysical quality of puncture data.



\medskip
\noindent
{\bf Black hole evolutions}
\smallskip

Once we have computed the initial data for a binary black hole system, we
are in principle set up to simulate their inspiral and merger. Unfortunately
however, this area of research is plagued by numerical instabilities that
have so far prevented long term simulations. For this reason I am very
interested in formulations of the Einstein equations with better stability
properties. First order symmetric hyperbolic formulations of the evolution
equations are promising in this respect, as they are guaranteed to yield
results which converge to the true solution at least for a finite time
interval. However, even if an evolution system is symmetric hyperbolic there
is no guarantee that its numerical implementation will be stable in the long
run when we evolve a highly dynamic black hole spacetime. Thus any system
of interest should be tested numerically before we can draw
definite conclusions about its stability.

Recently I have been involved in implementing and testing two new evolution
systems. The first one was devised by Alekseenko and Arnold. This system as
a whole is not symmetric hyperbolic, however it possesses a symmetric
hyperbolic subsystem. Together with N. Jansen and B. Br\"ugmann I have
implemented and tested the numerical stability of this system within the BAM
code~\cite{Jansen03}. Unfortunately, our numerical experiments show that it
is less stable than the well known Baumgarte-Shapiro-Shibata-Nakamura (BSSN)
system when we evolve black holes in gauges which enhance stability, at
least for the BSSN system.

B. Br\"ugmann and I are currently investigating several strongly hyperbolic
first order versions of the BSSN system. In particular, I have focused on a
system devised by Frittelli and Reula, which is the first order extension of
the BSSN system with the least number of variables. I have derived a
parametrized transformation of variables which for a particular parameter
choice brings the Frittelli-Reula system into a form that extends another
first order version of BSSN due to Alcubierre, Br\"{u}gmann, Miller and
Suen, in effect unifying the two systems~\cite{Tichy04}.
The transformation is designed
such that it does not alter the hyperbolic properties of the original
Frittelli-Reula system. I have also implemented this new system in the BAM
code. Preliminary numerical results indicate that the system is not as stable
as the original BSSN system. However our system has several free parameters
and it is possible that stability will improve once we find the 
appropriate choice of parameters.

I am also actively involved in binary black hole evolutions using the more
traditional BSSN system.  I am working on extending the lifetime of our
simulations by choosing a shift which brings the black hole binary into
corotating coordinates. I have also worked on new techniques to excise the
black hole interiors from the computational grid, so that we can better use
the new types of initial data described above, which tend to be such that
instabilities occur inside the black holes.

The simulations described so far were performed on a single computational
grid with uniform resolution. However in order to resolve the black holes a
certain minimum resolution is needed. Using this same resolution also away
from the black holes is wasteful in terms of computer resources, so that it
is for example impossible to put the outer grid boundary far from the black
holes. For this reason it I am intrigued by the idea of mesh refinement,
which allows several overlapping computational grids with different
resolutions. Together with B. Br\"ugmann, I am currently involved in an
effort to introduce mesh refinement into the BAM code.  Using several nested
grids with different resolutions we are starting to perform evolutions of a
single black hole, which seem to be just as stable as without mesh
refinement.







\medskip % \bigskip
\noindent
{\bf Coordinate independent formulation of post-Newtonian theory} 
\smallskip

In many problems of interest in astrophysics it is impossible to solve the
full Einstein equations of General Relativity analytically. For this reason
the post-Newtonian expansion in powers of velocity over the speed of light
has been developed. Usually, the equations of post-Newtonian theory are only
given in a specific coordinate system (or gauge). In each of the many
possible gauges the post-Newtonian equations take a different form, which
has the disadvantage that it is often hard to compare calculations in
different gauges. The situation is somewhat analogous to knowing
electrodynamics only in a couple of different gauges without knowing the
underlying gauge invariant Maxwell equations.
\'E. Flanagan and I have started investigating the possibility of
a gauge independent formulation of post-Newtonian General Relativity. At
first post-Newtonian order, we have found such a theory~\cite{Tichy03}. It
is formulated solely in terms of geometric objects, such as connections and
tensors. We are able to show from our equations that the usual
coordinate-dependent equations of post-Newtonian gravity can be recovered
when one specializes to specific coordinates.  
% It should also be possible to find higher than just post-1-Newtonian
% corrections by expanding General Relativity to higher orders.




\medskip % \bigskip
\noindent
{\bf Post-Newtonian gravitational waveforms }
\smallskip

The detection of gravitational waves from the inspiral of a compact binary
depends crucially on the availability of accurate template waveforms, as the
signal is likely to be buried in noise. If such a signal is to be detected
using matched filtering, high accuracy is needed in the phase of the
template. This phase is derived from the energy balance equation which
states that the energy loss rate is equal to the gravitational wave
luminosity of the binary. 
\'E. Flanagan, E. Poisson and I~\cite{Tichy:1999pv} have investigated
if the accuracy of the templates' phase can be improved by solving the
post-Newtonian energy balance equation exactly numerically, rather than (as
is normally done) solving the energy balance equation analytically within
the post-Newtonian perturbative expansion. We found evidence that there is
no gain in accuracy. This result is disappointing, but constitutes useful
information from the point of view of generating template banks for inspiral
searches: there is no motivation in terms of increased event rate to solve
numerically for the wave's phase.




\medskip % \bigskip
\noindent
{\bf Towards computing the radiation reaction force
        for test-particles in Kerr spacetime}
\smallskip

The inspiral of compact stellar mass objects into supermassive black holes
is an important source of gravitational waves which could be detected by the
planned space-based LISA detector. Since the mass ratio is small, the
compact object is well approximated by a test-particle. In order to find the
inspiral orbit of a test-particle due to radiation reaction, \'E. Flanagan
and I have tried to approximate the orbit as a sequence of geodesic orbits,
with constants of the motion that slowly change on a radiation reaction
timescale. The idea is to compute this adiabatic evolution of the constants
of the motion from quantities such as energy and angular momentum and their
fluxes at future null infinity. The first potential problem was that angular
momentum is not necessarily well defined. However, we find~\cite{Tichy00b}
that angular momentum ambiguities arise only at higher order in the
adiabatic expansions and can thus be neglected. A more serious problem is
that we have shown~\cite{Tichy00b} that the $x$- and
$y$-components of the angular momentum flux vanish identically in the
adiabatic approximation, so that energy and angular momentum alone do
not contain enough information to compute the evolution of all constants of
the motion for generic non-circular orbits around rotating (Kerr) 
black holes.





\medskip % \bigskip
\noindent
{\bf The expected stress-energy tensor of a massive scalar field }
\smallskip

In semiclassical relativity, a classical metric is coupled to quantum
fields. This is achieved by replacing the classical stress-energy tensor by
its expected value in the Einstein equation. A difficulty in semiclassical
relativity is the non-uniqueness of the expected stress-energy tensor. Wald
has postulated a set of physically well motivated axioms or criteria, which
any expected stress-energy tensor has to satisfy.  For a massive scalar
field \'E. Flanagan and I have shown that the ambiguity in the
stress-energy tensor allowed by the Wald axioms is an infinite parameter
ambiguity~\cite{Tichy:1998ws}, which is different from the two parameter
ambiguity known for massless scalar fields. In order to get an explicit
example, we have also calculated the expected stress-energy tensor in the
incoming vacuum state for a massive scalar field in a spacetime which is
a linear perturbation of Minkowski spacetime. Somewhat surprisingly, the
result of our specific calculation explicitly exhibits only a two parameter
ambiguity, just as in the massless case, even though this is not guaranteed
by the Wald axioms. 
It will be interesting to see if the Wald axioms can be
extended to rule out more than two undetermined parameters in the massive
case as well.









\bigskip
\bigskip

\clearpage

\noindent
\centerline{\bf \large Future Research Plans}
\medskip % \bigskip

\noindent
{\bf Astrophysically realistic initial data}
\smallskip

Black hole inspirals and mergers will be observable by gravitational wave
detectors. In the coming years numerical simulations will be used to predict
their highly relativistic orbits and waveforms. However, the astrophysical
relevance of numerical simulations depends crucially on the use of
astrophysically realistic initial data. There are at least two conditions
which the data have to fulfill in order to be realistic: (i) the data have
to agree with post-Newtonian data far from the black holes, (ii) near the
black holes they should agree with predictions from black hole perturbation
theory, in particular the black holes should have the correct amount of
tidal deformation. In addition it would be very useful to have the data in
coordinates, which are corotating with the black holes. Our post-Newtonian
data~\cite{Tichy02} already satisfy (i). I have several ideas on how to
verify the quality of the initial data near each horizon by comparing with
perturbed black hole results. For example, it will be important to
determine the tidal deformation of the horizon. Preliminary calculations
show that this tidal deformation should be very small for realistic data. It
may also be useful to use the isolated horizon framework, introduced by
Ashtekar and collaborators, to compute invariant quantities on the horizon.
Together with quantities computed at infinity this may open a way to
estimate the radiation content of the data. In this area there is a lot of
opportunities for collaboration with mathematical relativists.  I plan to
use these different ideas to develop a set of tests, which can be used to
assess the quality of initial data sets. I also want to use my knowledge
about post-Newtonian theory to match perturbed black hole solutions to
post-Newtonian solutions in order to obtain more realistic data. 
% This would
% also complement research on perturbed black holes already pursued at LSU.
Combining different perturbation methods with numerical calculations may
give valuable information about black hole binaries at intermediate
separations and will provide better initial data for fully numerical
simulations.

Another line of research I want to follow up on is to use the Cactus code
to evolve our post-Newtonian data~\cite{Tichy02}, and then to extract
waveforms using the Lazarus approach pioneered by Baker, Br\"ugmann,
Campanelli and Lousto. I have just started to collaborate on this project
with M. Campanelli at the Center for Gravitational Wave Astronomy in
Brownsville. The idea is that even if our post-Newtonian data is not yet
completely realistic in terms of astrophysical content, we want to get the
infrastructure ready to extract waves from numerically evolved
post-Newtonian data. At the same time I am also planning to add
post-Newtonian waveforms to our initial data. If we then evolve numerically
we might eventually be able to compute numerical waveforms, which
continuously match post-Newtonian waveforms.


\noindent
{\bf Binary Neutron stars with arbitrary spins}
\smallskip

As in the case of black holes, astrophysically realistic neutron stars are
expected to be spinning. In the case of milisecond a pulsar the
rotational velocity not too far from the mass shedding limit
which implies a dimensionless spin magnitude that could
be close to one. In a binary such spins would influence the
neutron star trajectories even before merger, and the merger
could also be influenced, e.g. if the merged object would have
too much angular momentum.
I propose to study the effects of high spins in neutron star
mergers. As already mentioned (in ???) my SGRID code can produce
neutron star initial data for corotating configurations. I intend
to extend this method to allow arbitrary spins. Once I have
initial data I plan to use the Whisky
code to evolve the two stars. This hydrodynamics 
code which is publically available 
and offers mesh refinement via the Carpet driver,
is ideally suited for this task.




\medskip % \bigskip
\noindent
{\bf Fully non-linear evolutions}
\smallskip

So far fully non-linear numerical evolutions of black holes in three
dimensions have always been hampered by numerical instabilities. I think
that it is a fascinating challenge to come up with improvements in this
area. I plan to continue working on symmetric hyperbolic evolution systems
for the Einstein equations in order to improve numerical stability.
% This
% would enhance research in this direction already carried out at LSU. 
In addition I am interested in finding gauge conditions which are well adapted
to the problem at hand, such as corotating coordinates for the binary black
hole problem. There are however several other ideas I wish to explore.
Einstein's equations naturally split into evolution equation and constraint
equations. Usually one solves the constraint equations for the initial data.
After that one only uses the evolution equations to evolve forward in time.
Analytically this is perfectly reasonable as the evolution equations
preserve the constraints. However numerical errors may cause the numerical
solution to wander away from constraint fulfillment, resulting in a solution
which does not satisfy all of Einstein's equations. Moreover such constraint
violations often grow exponentially and cause the computer code to fail. I
plan to investigate methods to suppress constraint violations. One possible
avenue is to use my experience in solving elliptic equations to re-solve the
constraints at least from time to time during the evolution. A second
possibility is to introduce additional variables, such that some of the
constraint equations become evolution equations for these new variables,
with the effect that these constraints are then automatically satisfied
during evolution. I think it is also possible to combine these two methods.
Another possibility is to use a new system of equations due to Bona,
Ledvinka, Palenzuela and Zacek, who extend Einstein's equations by
introducing additional fields. Of course in order to get a real physical
solution these fields have to vanish identically. The advantage however is
that this new system has no more constraint equations, only evolution
equations. The additional fields, which are closely related to the deviation
from the constraints in the original theory, obey wave equations, so that it
is straightforward to devise boundary conditions which prevent such fields
from entering the computational domain, while at the same time allowing them
to leave it. In this way one has more control over the constraint
deviations. I plan to use a combination of these methods to get a more
stable evolution system.

Finally, I am also interested in working on problems containing matter. For
example I would like to work on simulations of neutron star binaries or
binaries containing a black hole and a neutron star. At least in the latter
case, my experience with mesh refinement may be helpful because of the
difference in either size or field strength of the two objects.  Also, as
matter fields often develop shocks the evolution equations should be
expressed in flux-conservative form. Here my knowledge about symmetric
hyperbolic evolution may prove useful, as such systems can most easily be
expressed in flux-conservative form. Another area I am interested in working 
on in the future is the general relativistic simulation of stellar 
core collapses, and the formation of neutron stars. 
In both these areas I see the
potential for interdisciplinary cooperation with researchers in other areas
of physics or astronomy.


% I will be very excited to join LSU and to become part of one of the
% strongest Relativity groups in the world, but I think that I will also have
% an opportunity to develop my own identity at LSU. For example, I would very
% much enjoy to perhaps lead a small research group of students. Here at Penn
% State and also back at the Albert-Einstein-Institut, 
% I have always liked to interact or work with
% students. Also, I think that my experience with Cactus will help me to
% easily integrate into the numerical group at LSU. In addition, I am
% interested to learn more about grid computing, which also makes LSU a
% natural choice for me.

In closing I would like to remark that I am also open to new ideas
and willing to work in new areas of research.



\bigskip
\bigskip
\noindent
\centerline{\bf \large Commitment to Teaching}
\medskip % \bigskip

\noindent
{\bf Teaching experience }
\smallskip

While still at the University of Karlsruhe I was a teaching assistant for
graduate classes in classical and also quantum mechanics. My main task was
to highlight important concepts and to explain homework problems on the
blackboard to graduate students and to prepare solution for homework
problems. As a physics graduate student at Cornell University I was a
teaching assistant in many different classes, ranging from General Physics
classes for non-physics majors to Modern Physics classes for students in the
honors physics sequence. My main duties were to prepare material for the
classes I had to teach and to answer any questions the students may have
had. I also had to prepare and grade weekly or bi-weekly quizzes. In
addition I had office hours, where the students could ask questions. And of
course I had to grade exams and homeworks, and write up homework
solutions. I have also taught so-called Coop sessions where I divided the
students into groups to work on problems in teams. Similarly, I have taught
lab sessions where the students had to work on experiments in groups. In
either case I went from group to group to answer questions, to help, or to
ask questions myself in order to see if the students understood important
concepts. Also at Cornell I was a grader in a statistical mechanics class
taught by Ashcroft. My task was to grade the student's exams and homeworks
and to write up solutions for the homework problems to be handed out to the
students. All this has given me a lot of first hand experience on how
students learn and also on what may prevent them from learning. I am eager
to apply this experience in the classroom.


\medskip
\noindent
{\bf Teaching philosophy}
\smallskip

A good teacher should recognize individual differences among students,
encourage students to evolve and develop their own ideas, motivate them and
recognize their progress. In addition to conducting lectures
and designing labs the teacher should be available for the students during
office hours. They should feel free and welcome to ask questions and discuss
matters concerning them.

In teaching it is essential that the students are motivated to
be curious about the material. Otherwise it is very hard for them to learn.
In order to motivate students, lectures should be interesting
and structured such that the students can see that they are making progress.
It is very important to explain the basic concepts first and to explain them
well, before embarking on long lectures about calculational methods.
I think it is better to explain one thing well instead of explaining a lot
of different things superficially. 
This way the students better understand the material and
will be more motivated. Also it is important to pause from time to time to
give them a chance to ask questions, and if they do not have questions,
it may be good to ask them a question.
Another crucial point is to have the lecture well planned. I plan to make
and hand out notes about everything I wish to cover.

Furthermore it is important to take the educational background of the
students into account.  Especially their mathematical background should be
considered in teaching physics. For example if most of the students are not
familiar with differential equations, it is probably not helpful to use such
equations, unless there is enough time to thoroughly explain the necessary
mathematical background.

Assigned homework should be carefully selected to illustrate the concepts or
calculational methods, explained in lecture. I think homeworks should count
toward the final grade so that the students are motivated to do them. 
Nevertheless, I think it is good to not swamp the students with work. 
The amount of homework should be chosen such that they have time to 
think and read about the material. Less is sometimes more.

I think that doing experiments in physics is helpful, especially if they are
fun. The hands-on experience may motivate students. Yet, the experiments
should be designed such that they only require knowledge of material already
covered in lecture. I do not think that it is a good idea to introduce new
concepts in a lab. The students should enjoy doing the experiments, in order
to motivate them to understand the material. In no case should their time in
the lab be wasted. Once I saw an experiment in an introductory optics
class, where the students first had to spend an hour adjusting mirrors and
lenses in order to see some hydrogen lines. Such experiments are
demotivating and do not really teach the material they intend to teach.


\medskip
\noindent
{\bf Teaching interests}
\smallskip

At the graduate level, I would like to teach classes about General
Relativity, numerical methods, quantum mechanics or mathematical methods for
physicists. At the intermediate level I would like to teach classes in
modern physics, such as about particles and waves, quantum mechanics,
special relativity or electrodynamics. Of course I am also happy to teach
any physics class at the introductory level.






%\clearpage

\bibliographystyle{/usr/share/texmf/bibtex/bst/base/unsrt.bst}
%\bibliographystyle{/usr/share/texmf/bibtex/bst/base/abbrv.bst}
%\bibliographystyle{/usr/share/texmf/bibtex/bst/natbib/abbrvnat.bst}
%\bibliographystyle{apsrev}
%\bibliographystyle{/usr/share/texmf/tex/latex/revtex/seg.bst}
%\bibliographystyle{/usr/share/texmf/bibtex/bst/ams/amsplain.bst}


\bibliography{references}


\end{spacing}
\end{document}

